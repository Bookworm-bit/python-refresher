\documentclass[12pt]{article}

\usepackage{amsmath}			% A library of many standard math expressions
\usepackage[margin=1in]{geometry}% Sets 1in margins. 
\usepackage{fancyhdr}			% Creates headers and footers
\usepackage{enumerate}          %These two package give custom labels to a list
\usepackage[shortlabels]{enumitem}


% Creates the header and footer. You can adjust the look and feel of these here.
\pagestyle{fancy}
\fancyhead[l]{Eric Zhang}
\fancyhead[c]{BWSI AUVC - 1.5 Calculus \#1}
\fancyhead[r]{\today}
\fancyfoot[c]{\thepage}
\renewcommand{\headrulewidth}{0.2pt}
\setlength{\headheight}{15pt}



\begin{document}

\noindent Find the derivative of the following functions

\begin{enumerate}[start=1,label={\bfseries Question \arabic*:},leftmargin=1in]
    \item $f(x) = x^2 + 3x - 5$
    \item $g(x) = \sin(x) + \cos(x)$
    \item $h(x) = \frac{1}{x^3} + \tan(x)$
    \item $k(x) = \cos^2(x^2)$
\end{enumerate}

\begin{enumerate}[start=1,label={\bfseries Solution \arabic*:},leftmargin=1in]
    \item Trivially, $f'(x) = 2x + 3$
    \item Trivially, $g'(x) = \cos(x) - \sin(x)$
    \item Recalling $\frac{\mathrm{d}}{\mathrm{d}x} \tan(x) = \frac{1}{\cos^2(x)}$ and rewriting $\frac{1}{x^3} = x^{-3}$, $h'(x) = -3x^{-4} + \frac{1}{\cos^2(x)}$
    \item Repeatedly using chain rule, $k'(x) = 2\cos(x^2) \cdot \frac{\mathrm{d}}{\mathrm{d}x} (\cos(x^2)) = 2\cos(x^2) \cdot -\sin(x^2) \cdot 2x = -4x\cos(x^2)\sin(x^2)$
\end{enumerate}

\noindent \\Evaluate the following integrals

\begin{enumerate}[start=1,label={\bfseries Question \arabic*:},leftmargin=1in]
    \item $\int (x^2 + 3x - 5) \mathrm{d}x$
    \item $\int (\sin(x) + \cos(x)) \mathrm{d}x$
    \item $\int (\frac{1}{x^3} + \cos(x)) \mathrm{d}x$
\end{enumerate}

\begin{enumerate}[start=1,label={\bfseries Solution \arabic*:},leftmargin=1in]
    \item Trivially, the integral is split and evaluates to $\frac{1}{3} x^3 + \frac{3}{2} x^2 - 5x + C$
    \item Trivially, the integral evaluates to $-\cos(x) + \sin(x) + C$
    \item Rewriting $\frac{1}{x^3} = x^{-3}$, the integral evaluates to $\frac{x^{-2}}{-2} + \sin(x) + C$
\end{enumerate}

\end{document}
